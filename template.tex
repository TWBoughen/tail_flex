% Options for packages loaded elsewhere
\PassOptionsToPackage{unicode}{hyperref}
\PassOptionsToPackage{hyphens}{url}
\PassOptionsToPackage{dvipsnames,svgnames,x11names}{xcolor}
%
\documentclass[
  sn-nature,
]{sn-jnl}



\usepackage{amsmath,amssymb}
\usepackage{iftex}
\ifPDFTeX
  \usepackage[T1]{fontenc}
  \usepackage[utf8]{inputenc}
  \usepackage{textcomp} % provide euro and other symbols
\else % if luatex or xetex
  \usepackage{unicode-math}
  \defaultfontfeatures{Scale=MatchLowercase}
  \defaultfontfeatures[\rmfamily]{Ligatures=TeX,Scale=1}
\fi
\usepackage{lmodern}
\ifPDFTeX\else  
    % xetex/luatex font selection
\fi
% Use upquote if available, for straight quotes in verbatim environments
\IfFileExists{upquote.sty}{\usepackage{upquote}}{}
\IfFileExists{microtype.sty}{% use microtype if available
  \usepackage[]{microtype}
  \UseMicrotypeSet[protrusion]{basicmath} % disable protrusion for tt fonts
}{}
\makeatletter
\@ifundefined{KOMAClassName}{% if non-KOMA class
  \IfFileExists{parskip.sty}{%
    \usepackage{parskip}
  }{% else
    \setlength{\parindent}{0pt}
    \setlength{\parskip}{6pt plus 2pt minus 1pt}}
}{% if KOMA class
  \KOMAoptions{parskip=half}}
\makeatother
\usepackage{xcolor}
\setlength{\emergencystretch}{3em} % prevent overfull lines
\setcounter{secnumdepth}{-\maxdimen} % remove section numbering
% Make \paragraph and \subparagraph free-standing
\ifx\paragraph\undefined\else
  \let\oldparagraph\paragraph
  \renewcommand{\paragraph}[1]{\oldparagraph{#1}\mbox{}}
\fi
\ifx\subparagraph\undefined\else
  \let\oldsubparagraph\subparagraph
  \renewcommand{\subparagraph}[1]{\oldsubparagraph{#1}\mbox{}}
\fi


\providecommand{\tightlist}{%
  \setlength{\itemsep}{0pt}\setlength{\parskip}{0pt}}\usepackage{longtable,booktabs,array}
\usepackage{calc} % for calculating minipage widths
% Correct order of tables after \paragraph or \subparagraph
\usepackage{etoolbox}
\makeatletter
\patchcmd\longtable{\par}{\if@noskipsec\mbox{}\fi\par}{}{}
\makeatother
% Allow footnotes in longtable head/foot
\IfFileExists{footnotehyper.sty}{\usepackage{footnotehyper}}{\usepackage{footnote}}
\makesavenoteenv{longtable}
\usepackage{graphicx}
\makeatletter
\def\maxwidth{\ifdim\Gin@nat@width>\linewidth\linewidth\else\Gin@nat@width\fi}
\def\maxheight{\ifdim\Gin@nat@height>\textheight\textheight\else\Gin@nat@height\fi}
\makeatother
% Scale images if necessary, so that they will not overflow the page
% margins by default, and it is still possible to overwrite the defaults
% using explicit options in \includegraphics[width, height, ...]{}
\setkeys{Gin}{width=\maxwidth,height=\maxheight,keepaspectratio}
% Set default figure placement to htbp
\makeatletter
\def\fps@figure{htbp}
\makeatother

%%%% Standard Packages

\usepackage{graphicx}%
\usepackage{multirow}%
\usepackage{amsmath,amssymb,amsfonts}%
\usepackage{amsthm}%
\usepackage{mathrsfs}%
\usepackage[title]{appendix}%
\usepackage{xcolor}%
\usepackage{textcomp}%
\usepackage{manyfoot}%
\usepackage{booktabs}%
\usepackage{algorithm}%
\usepackage{algorithmicx}%
\usepackage{algpseudocode}%
\usepackage{listings}%

%%%%

\raggedbottom
\makeatletter
\@ifpackageloaded{caption}{}{\usepackage{caption}}
\AtBeginDocument{%
\ifdefined\contentsname
  \renewcommand*\contentsname{Table of contents}
\else
  \newcommand\contentsname{Table of contents}
\fi
\ifdefined\listfigurename
  \renewcommand*\listfigurename{List of Figures}
\else
  \newcommand\listfigurename{List of Figures}
\fi
\ifdefined\listtablename
  \renewcommand*\listtablename{List of Tables}
\else
  \newcommand\listtablename{List of Tables}
\fi
\ifdefined\figurename
  \renewcommand*\figurename{Figure}
\else
  \newcommand\figurename{Figure}
\fi
\ifdefined\tablename
  \renewcommand*\tablename{Table}
\else
  \newcommand\tablename{Table}
\fi
}
\@ifpackageloaded{float}{}{\usepackage{float}}
\floatstyle{ruled}
\@ifundefined{c@chapter}{\newfloat{codelisting}{h}{lop}}{\newfloat{codelisting}{h}{lop}[chapter]}
\floatname{codelisting}{Listing}
\newcommand*\listoflistings{\listof{codelisting}{List of Listings}}
\makeatother
\makeatletter
\makeatother
\makeatletter
\@ifpackageloaded{caption}{}{\usepackage{caption}}
\@ifpackageloaded{subcaption}{}{\usepackage{subcaption}}
\makeatother
\ifLuaTeX
  \usepackage{selnolig}  % disable illegal ligatures
\fi
\usepackage{bookmark}

\IfFileExists{xurl.sty}{\usepackage{xurl}}{} % add URL line breaks if available
\urlstyle{same} % disable monospaced font for URLs
\hypersetup{
  pdftitle={Quarto Template for Springer Nature},
  pdfauthor={Author One; Author Two; Author Three},
  pdfkeywords={template, demo},
  colorlinks=true,
  linkcolor={blue},
  filecolor={Maroon},
  citecolor={Blue},
  urlcolor={Blue},
  pdfcreator={LaTeX via pandoc}}

\title[Quarto Template for Springer Nature]{Quarto Template for Springer
Nature}

% author setup
\author[1,2]{\fnm{Author} \sur{One}}\equalcont{These authors contributed equally to this work.}\author*[3]{\fnm{Author} \sur{Two}}\email{corresponding@email.com}\equalcont{These authors contributed equally to this work.}\author[1]{\fnm{Author} \sur{Three}}
% affil setup
\affil[1]{\orgdiv{Department of Government}, \orgname{Harvard
University}, \orgaddress{\street{1737 Cambridge
Street}, \city{Cambridge}, \postcode{2138}}}
\affil[2]{\orgdiv{Department of Statistics}, \orgname{Harvard
University}, \orgaddress{\street{1 Oxford
Street}, \city{Cambridge}, \postcode{2138}}}
\affil[3]{\orgdiv{Department of Political Science}, \orgname{Yale
University}, \orgaddress{\street{115 Prospect Street}, \city{New
Haven}, \postcode{3401}}}

% abstract 

\abstract{The abstract serves both as a general introduction to the
topic and as a brief, non-technical summary of the main results and
their implications. Authors are advised to check the author instructions
for the journal they are submitting to for word limits and if structural
elements like subheadings, citations, or equations are permitted.}

% keywords
\keywords{template,  demo}

\begin{document}
\maketitle

\section{Introduction}\label{sec-intro}

The Introduction section, of referenced text \citep{greenwade93} expands
on the background of the work (some overlap with the Abstract is
acceptable). The introduction should not include subheadings.

Springer Nature does not impose a strict layout as standard however
authors are advised to check the individual requirements for the journal
they are planning to submit to as there may be journal-level
preferences. When preparing your text please also be aware that some
stylistic choices are not supported in full text XML (publication
version), including coloured font. These will not be replicated in the
typeset article if it is accepted.

\section{Using this template}\label{sec-template}

This Quarto template is unofficial and built out of Springer Nature's
template. Some examples of commonly used commands and features are
listed below, to help you get started.

As seen below, you can mix markdown and Latex with each other, though
it's likely best to mostly use markdown.

\subsection{Cross Referencing}\label{cross-referencing}

To reference a figure with example label ``plot'', use
\texttt{@fig-plot}. Analogously, to reference a table with example label
``data'', use \texttt{@tbl-data}. To reference a section, such as the
Introduction (Section~\ref{sec-intro}), use \texttt{@sec-intro}.

For complete information on cross referencing with Quarto, see
\url{https://quarto.org/docs/authoring/cross-references.html}.

\subsection{Citations}\label{citations}

Quarto formats citations and references automatically using the
bibliography records in your .bib file. For a citation in parentheses
use \citep{greenwade93}. Multiple citations can be given as
\citep{greenwade93, knuth1984texbook}. If the tex output is to be
included in a submission to a preprint server or publisher, the default
citation method, \texttt{citeproc}, may not always produce
bibliographies compatible with the provided \texttt{bst} styles. In this
case, you can set \texttt{cite-method:\ natbib}. See
\url{https://quarto.org/docs/authoring/footnotes-and-citations.html\#sec-biblatex}.
In this case, you may either restrict usage to the default bracketed key
citation formats supported by pandoc, such as \citep{greenwade93} or
\citep{greenwade93, knuth1984texbook}. Or, use a CSL file, several of
which are vendored in
\href{https://github.com/christopherkenny/nature/tree/main/_extensions/nature/csl}{\_extensions/nature/csl}.
The advantage of the latter is that references will be consistent across
html, docx, and pdf outputs. See
\url{https://pandoc.org/MANUAL.html\#citation-syntax} for additional
details on pandoc citation styles.

\section{Tables and Figures}\label{tables-and-figures}

To include figures, you can use Quarto syntax.

\begin{figure}

\centering{

\includegraphics{fig.eps}

}

\caption{\label{fig-plot}An example figure (an empty plot)}

\end{figure}%

For both figures and tables, you can use LaTeX syntax if you need
additional customizability.

For example, to use footnotes within a table, you may want to use LaTeX.

\begin{table}

\caption{\label{tbl-data}LaTeX caption text}

\centering{

[h]
%
\begin{tabular}{@{}llll@{}}
\toprule
Column 1 & Column 2  & Column 3 & Column 4\\
\midrule
row 1    & data 1   & data 2  & data 3  \\
row 2    & data 4   & data 5\footnotemark[1]  & data 6  \\
row 3    & data 7   & data 8  & data 9\footnotemark[2]  \\
\botrule
\end{tabular}
\footnotetext{Source: This is an example of table footnote. This is an example of table footnote.}
\footnotetext[1]{Example for a first table footnote. This is an example of table footnote.}
\footnotetext[2]{Example for a second table footnote. This is an example of table footnote.}

}

\end{table}%

\section{References}\label{references}

\renewcommand{\bibsection}{}
\bibliography{bibliography.bib}

\section{Appendix}\label{appendix}






\end{document}
